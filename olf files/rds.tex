\documentclass[12pt]{report}
\begin{document}

\chapter{Respondent-driven sampling}
\subsection{copied}
Respondent-driven sampling is a technique for estimating traits in hidden population. It is widely used for studying prevalence of HIV/AIDS among injection drug users, sex workers, men who have sex with men.  

Studying prevalence of disease can help to understand and control its spreading. Unfortunately, there are difficulties with such kind of research as there is no sampling frame and members of hidden groups may not want reveal themselves. 

There are several existing solutions for sampling hidden population such as snowball sampling, targeted sampling, time-space sampling, key-informant sampling. The main disadvantage of all these methods is unknown bias and variance of obtained estimation.

%Respondent-driven sampling (RDS) is a technique for estimation needed characteristics in hidden populations. With this technique it is possible to collect data from the populations that are hard to reach.

RDS begins with selecting group of initial participants that are called seeds. The procedure follows according to chain-referral model: each participant in study recruits another participants. The step is called wave. Both participating and recruiting new participants are encouraged by financial incentive. The sampling continues in this way until needed size of participants is reached. During RDS participants are asked to report how many contacts they have.
This process enables to collect data for making statistical analysis.

In order to study formally RDS can be regarded as Markov Chain.
Assumptions:
\begin{enumerate}
  \item Seeds are chosen proportionally to their degree in the network.  
  \item If individual $A$ knows individual $B$ than individual $B$ knows $A$ as well (network can be represented as undirected graph).
  \item The same individual can be recruited multiple times (sampling with replacement.
  \item The choice of contacts to recruit is uniformly at random.
  \item Individuals know precisely their network degree. 
  \item Each individual is reachable from each other individual (network is connected). 
\end{enumerate}


For this process stationary distribution is exactly distribution proportional to network degree. So first assumption guaranties that not only first but all samples during the process are taken with probability proportional to the degree of participants in the network. In \cite{salganik2004sampling} this assumption is considered to be reasonable as the people that are drawn as seeds are well-known people and they have usually more contacts than on average. Without this assumption first there should be performed enough number of waves until sample can be considered drawn from stationary distribution.
 simulation studies about assumptions violation(sensitivity) \cite{gile2010respondent}

studies of variance

In this way individuals with more friends (contacts) are more likely to be recruited. To correct this bias the responses from individuals are weighted according to their degree (number of contacts).
Let $X_1, X_2, ..., X_n$ be all collected samples during RDS.
Then estimate $\mu_f$ of the population mean of $f$ is defined \cite{goel2010assessing} as
$$\mu_f = \frac{1}{\sum\limits_{i=1}^n 1/degree(X_i)} \sum\limits_{i=1}^n \frac{f(X_i)}{degree(X_i)}$$

RDS can perform poorly if the groups of individuals form different communities. It is known fact that friends tend to have similar traits. This fact becomes a source of bias in chain-referral methods of sampling. Structure of network also affects a lot. In \cite{goel2009respondent} it is shown that 'bottlenecks' between different groups in hidden population increases variance of RDS estimator. They try RDS on network structure with communities, but where individuals, that are in contact with each other, do not have similar traits and showed that such structure indeed affects on RDS estimate. 

Design effect $d$ is variance of RDS estimate over variance of estimate obtained from simple random sampling (SRS). It means that if for SDS we need $n$ samples than to have RDS estimate with the same variance we need $dn$ samples. 

It is known fact that people tend to be friends if they share some traits: have similar age, common language, the same university.

Homophily - the tendency for individuals with similar attributes to be friends with one another.
% not mine
The fact that the majority of participants are recruited by other respondents and not by researchers makes RDS a successful method of data collection. However, the same feature also inherently complicates inference because it requires researchers to make assumptions about the recruitment
process and the structure of the social network connecting the study population.

\bibliographystyle{plain}
\bibliography{rds}

\end{document}