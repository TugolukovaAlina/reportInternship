\documentclass[12pt]{report}
\usepackage{graphicx}
\usepackage{caption}
\usepackage{subcaption}

\graphicspath{ {images/} }

\begin{document}


\section{put in the report}

In order to make correct estimates it is not enough to have just subset of individuals. We also have to know the probability of one particular individual to be selected. For example, by using telephone survey in order to collect information we automatically exclude some subsets of people (like homeless, poor) which can affect the correctness of the estimate because it is impossible to predict bias. 

There are several existing solutions for sampling hidden population such as snowball sampling, targeted sampling, time-space sampling, key-informant sampling. The main disadvantage of all these methods is unknown bias and variance of obtained estimation.

We are living in the era of information when it is crucial to collect data, to be able to analyze them and draw potentially valuable conclusions. Particularly it is interesting to analyze network structures such as online-social networks (OSNs), peer-to-peer networks (P2P) or network of individuals.

There can be variety reasons to collect information about the networks.

For example, we can be interested in estimation of total number of peers in network or number of peers that satisfy needed characteristics. This information can be used in peer-to-peer protocols. For example, peer-to-peer protocol Viceroy needs to know number of nodes in network before including the new one in it (2). Some gossip based peer-to-peer protocols require knowledge about network size in order to disseminate information (2).

OSNs possess huge amount of information about population that can be interesting for different areas of life: sociology, marketing, network engineering (3).

Another example it is human networks.

Unfortunately sampling such kind of structure is not always evident and easy. It is not always possible to identify all the nodes of the network in order to take representative subset of them for the analysis. 


The simplest idea is to take node uniformly at random knowing the identity of all nodes in network (uniform sampling). This technique can provide us uniform choosing of nodes and independency of received samples. But here we can confront some problems.

Having all these advantages of P2P networks, on the other side, it is not so easy to collect needed characteristics of network what is direct in centralized systems. Moreover, the P2P networks have distributed nature, what usually implies that no node maintains the knowledge of all topology. Nevertheless, even if P2P protocol assumes existing of such a node (like BitTorrent tracker in BitTorrent protocol) it is usually regarded as its weak side. 

In social networks each user has ID. So having the whole list of IDs would perfectly fit to uniform sampling technique. However, the social network owners can hide information about all IDs due to their privacy policy. Moreover, some of the IDs can be not valid. 

Performing too much requests can be expensive in the meaning of resources (4). Rather than trying to find valid ID by random requests it can be more useful to choose small but representative set of nodes (3).

The other sampling techniques are based on random walking (crawling techniques). The network is regarded as a graph. The simplest method is called the Random Tour method, where probability to go from the node to each of his neighbor is equal. It is can be shown that probability distribution is not uniform. It is biased toward the nodes that have greater number of neighbors. 

The other methods remove this bias by spending less time in the nodes with greater number of neighbors. Particularly, we will regard three methods: Maximum degree 
method, Local degree method and Metropolis Hasting method. All crawling techniques work only on connected graphs while uniform sampling techniques can be applied even to disconnected. They also suffer from dependency of samples.

Though this network structure brings difficulties at the same time it can (naturally suggest) help to collect data from the network using chain-referral methods.
The one of such examples is sampling hidden populations(e.g., drug users).
Being comfortable method for finding people for studies RDS introduces some additional difficulties comparing to simple random sampling. The most important is dependency of the samples.

Then problem how to know variance (how to be able to say about confidence that result is correct).

RDS is introduced by Heckathorn 1997.

Many papers try to warn rds users.
Some paper are studying how the violation of some assumptions can influence on the RDS estimates.

In \cite{gile2010respondent} is studied  the sensitivity to the procedure for selecting the seeds, sensitivity to respondent behavior, sensitivity to the with-replacement assumption.

In order to study we have random graphs where we have some parameters to change. We want the same with values.



IT is written in \cite{gile2010respondent} that V-H estimator outperforms S-H estimator.


Show that energy is less then in random configuration 

[try to sample independenlty but with probabilities pi and look at the variance. is it the same, i think no].



!!! compare when reported degree is not correct and using just mean


$B$ - budget

$C_1$ - cost of one step of walk (individuals just provide the correct number of their contacts)

$C_2$ - cost of participation (cost of interview with individuals)

$n$ - number of steps

$m$ - number of participants from $n$

The next equality should be true:

$$B = n \cdot C_1 + m \cdot C_2$$

If we want to skip $k$ steps between taking the node as a participant then
$$B = nC_1 + \frac{n}{k+1}C_2$$

Here $m = \frac{n}{k+1}$ as we take each $k+1$ node as a participant. So having budget $B$ and skipping each $k$ node allows as to perform $n = \frac{(k+1)B}{(k+1)C_1 + C_2}$ steps with $m = \frac{B}{(k+1)C_1 + C_2}$ number of participants.

Then variance:

$$\frac{\sigma^2}{\frac{B}{(k+1)C_1 + C_2}} \frac{1+\rho^{k+1}}{1-\rho^{k+1}}$$ 



Then 
$P^*$ is symmetric [explain why]
$P^*$ is diagonalizable [explain why]
eigenvalues of $P$ and $P^*$ are the same


Function: variance of mean with skipping having fixed budget and payments (general case)
$$ var\left[\bar{X} \right] = \frac{1}{n} \sum_{i=2}^r\frac{1-\lambda_i^2 - 2\frac{\lambda_i}{n} + 2\frac{\lambda_i^{n+1}}{n}}{(1 - \lambda_i)^2} <f, v_i>^2_{\pi}$$



$$ \sigma_{\hat{\mu}}^2(k) = \frac{\sigma^2}{\frac{B}{kC_1 + C_2}} \frac{1+\rho^{k}}{1-\rho^{k}}$$.

$$  \sigma_{\hat{\mu}}^2(k) = \frac{1}{\frac{B}{kC_1 + C_2}} \sum_{i=2}^r\frac{1 + \lambda_i^k}{1 - \lambda_i^k} <g, v_i>^2_{\pi}$$



$$\sigma_{\hat{\mu}}^2 = \frac{\sigma^2}{n} \frac{1+\rho}{1-\rho}$$

$$\sigma_{\hat{\mu}}^2(k) = \frac{\sigma^2}{n} \frac{1+\rho^k}{1-\rho^k}$$


\end{document}